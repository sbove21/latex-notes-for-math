\documentclass{article}
\usepackage{graphicx} % Required for inserting images

\title{[SEC-5.3] Polynomials and Polynomial Functions}
\author{Sam Bove}
\date{\today}

\begin{document}

\maketitle

\section{Polynomial Vocabulary}
\begin{itemize}
    \item \textbf{Term} - A number or product of a number and variables raised to powers \\ EX: $a^2 a^4 etc...$
    \item \textbf{Coefficient} - Numerical factor of a term \\ EX: 10 would be the coefficient of $10a$
    \item \textbf{Constant} - Term which is only a number \\ EX: 10, 5, 4, 100, etc
    \item \textbf{Polynomial} - A sum of terms involving variables raised to a whole number exponent with no variables appearing in any denominator     
\end{itemize}
\section{Polynomial breakdown example}
Lets breakdown the polynomial $8x^5 + x^2y^2 - 4xy + 7$ \\ We have four terms here:
\begin{enumerate}
    \item $8x^5$
    \item $x^2y^2$
    \item $-4xy$
    \item $7$
    \end{enumerate}
Our coefficients are:
\begin{itemize}
    \item 8 is the coefficient of $8x^5$
    \item 1 is the coefficient of $x^2y^2$
    \item -4 is the coefficient of $-4xy$
\end{itemize}
And we only have one constant term which in this case is 7
\section{Types of polynoimials}
\begin{itemize}
    \item Monomial: is a polynomial with exactly one term
    \item Binomial: is a polynomial with exactly two terms
    \item Trinomial: is a polynomial with exactly three terms
\end{itemize}
\section{Degree of a term}
To find the degree of a term lets use $5a^4b^3c$ we would add the exponents together which in this case would be 8. 4+3 is 7 but we need to factor in that c although it doesn't show it is technically raised to the 1st power which would be 4+3+1 which = 8
\section{Degree of a polynomial}
Find the degree of all of the terms within a polynomial and take the largest degree out of all of those terms. That is your degree of the polynomial. 

\section{Adding Polynomials}
Its simply just as easy as combining like terms:
$(8y^3-4y^2+5) + (5y^2+1)$ \\ Combine $-4y^2$ and $5y^2$ to get $y^2$ \\ Combine 5 and 1 to get 6 \\ $8y^3$ doesn't combine with anything so it stays. \\ Our result is: $8y^3 + y^2 + 6$

\section{Subtracting Polynomials}
Take whatever the second polynomial is and just distribute a negative across all of the terms. We can then simply just add the two polynomials together.
\end{document}
