\documentclass[12pt, letterpaper]{article}

\title{Simplifying Exponents \textit{With Common Bases}}
\author{Samuel Bove}
\date{\today}





\begin{document}
	
	
	\maketitle


	\section{Introduction}
	The goal of this document is to serve as a personal reference to understand how exponents \emph{(with common bases),} can be simplified. This document also serves as my first ever \emph{LaTeX} creation. I am very excited to continue using this tool to aid in my mathematical endeavors.
	
	\section{The Basic Simplification Techniques}
	
	\centering
	\textbf{The product rule:}
	
	If \textit{y} and \textit{z} are integers and \textit{x} is a real number, then:
	
	$x^y \cdot x^z = x^{y+z}$
	
	\textbf{The quotient rule:}
	
	If \textit{x} is a nonzero real number and \textit{y} and \textit{z} are integers, then:
	
	$\frac{x^y}{x^z} = x^{y-z}$
	
	\textbf{Negative Exponents:}
	
	If \textit{x} is a real number other than 0 and \textit{y} is a positive integer, then:
	
	$a^{-n}=\frac{1}{a^n}$ \\ $\frac{1}{a^{-n}}=\frac{a^n}{1}$ or $a^n$
	
	\textbf{Zero Exponent:}
	
	$a^0 = 1$
	
	
	
	
	
	
	

	
	
	
	
	
\end{document}