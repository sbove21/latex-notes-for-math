\documentclass[12pt, letterpaper]{article}

\title{Simplifying Exponents \textit{With Common Bases}}
\author{Samuel Bove}
\date{\today}





\begin{document}
	
	
	\maketitle


	\section{Introduction}
	The goal of this document is to serve as a personal reference to understand how exponents \emph{(with common bases),} can be simplified. This document also serves as my first ever \emph{LaTeX} creation. I am very excited to continue using this tool to aid in my mathematical endeavors.
	
	\section{The Basic Simplification Techniques}
	
	\begin{center}
	\textbf{The product rule:}
	
	If \textit{y} and \textit{z} are integers and \textit{x} is a real number, then:
	
	$x^y \cdot x^z = x^{y+z}$
	
	\textbf{The quotient rule:}
	
	If \textit{x} is a nonzero real number and \textit{y} and \textit{z} are integers, then:
	
	$\frac{x^y}{x^z} = x^{y-z}$
	
	\textbf{Negative Exponents:}
	
	If \textit{x} is a real number other than 0 and \textit{y} is a positive integer, then:
	
	$a^{-n}=\frac{1}{a^n}$ 
	
	$\frac{1}{a^{-n}}=\frac{a^n}{1}$ or $a^n$
	
	\textbf{Zero Exponent:}
	
	$a^0 = 1$
	
		
	\section{The three power rules}
	
	If $x$ and $y$ are real numbers and $a$ and $b$ are integers, then:
	
	\textbf{Power Rule:}
	
	$(x^a)^b = x^{ab}$
	
	\textbf{Power of a Product:}
	
	$(xy)^a = x^a \cdot y^a$
	
	\textbf{Power of a Quotient:}
	
	$\left( \frac{x}{y} \right)^a = \frac{x^a}{y^a}$

	\end{center}
	
	\section{Tips/Reminders}
	
	\begin{itemize}
		\item Even if the bases are the same, they cannot be combined if the exponents are not also the exact same. 
	\end{itemize}
\end{document}